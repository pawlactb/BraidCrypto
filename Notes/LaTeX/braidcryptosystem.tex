\documentclass{article}

\usepackage[margin=1in]{geometry}
\usepackage{amsmath}
\usepackage{amssymb}
\usepackage{amsthm}
\usepackage{graphicx}
\theoremstyle{definition}
\newtheorem{definition}{Definition}[section]
 
\title{Braid Group Cryptography}
\author{
  Madsen, Reese\\
  \texttt{madsenar@clarkson.edu}
  \and
  Pawlaczyk, Tyler\\
  \texttt{pawlactb@clarkson.edu}
   \and
  Klee, Bryan\\
  \texttt{kleebm@clarkson.edu}
}
\date{\today}


\begin{document}

	\maketitle

	\begin{abstract}
	\textbf{Write abstract after paper is done.}
	\end{abstract}
	\newpage
	\section{Braids}
	In this section we will explain the mathematics behind a braid group. A braid group has braids as the set and concatenation as the group operation written as $<B_n,||>$ where $n$ is the number of strands and 

\begin{definition} 
$B_n=\{\sigma_1,...,\sigma_{n-1}:\sigma_i\sigma_j\sigma_i=\sigma_j\sigma_i\sigma_j\ \textrm{if}\ |i-j|=1\ \textrm{and}\ \sigma_i\sigma_j=\sigma_j\sigma_i\ \textrm{if}\ |i-j|>1\}$
\end{definition}
 
A braid group is infinite and nonabelian meaning that the elements do not commute such that:
$a,b \in B_n: \ ab\neq ba$. Note that for $n$ strands there are $n-1$ generators represented by $\sigma$. A braid is the concatenation of generators. A positive generator, $\sigma_i^+$, corresponds to crossing left over right and a negative generator corresponds to crossing right over left, $\sigma_i^-$. 

\begin{figure}[hbp!]
\centering
  \includegraphics[width=0.2\linewidth]{./Pictures/gen_pos_neg.png}
  \caption{$\sigma_i^+$,$\sigma_i^-$}\label{fig:graph}
\end{figure}


For example the braid $b \in B_3: \ b=\sigma_2^+ \sigma_1^+ \sigma_2^- \sigma_1^-$ is the following:
	
	\begin{figure}[hbp!]
\centering
  \includegraphics[width=0.2\linewidth]{./Pictures/example_braid.png}
\end{figure}
	
	
\section{Braids as Permutations}
	There is no unique way to write a braid as a concatenation of generators. There is no unique form because you can simply use the rules of the braid group to switch generators around. Therefore there is a connection defined between braid groups and permutation groups so that we can define a uniqueness. There exists a endomorphism:
	$ \phi: B_n \rightarrow \Sigma_n$.
	$\sigma_i $ represents the switch of the i-th strand with the (i+1) strand. 
	Denote the permutation by $\pi \in \Sigma_n, \pi(i)=b_i$. The trick is to then draw straight lines for the permutations on the braid diagram. For example the permutation $\pi=(1 2 3 4)\rightarrow(4 2 1 3)$ maps the 1st strand to the 4th strand and so forth. This corresponds to the following braid $A=\sigma_1 \sigma_2 \sigma_1 \sigma_3$. The fundamental braid $\Delta_n=(\sigma_1...\sigma_{n-1})(\sigma_1 ... \sigma_{n-2})...\sigma_1 $ corresponds to the permutation $\Omega_n=n(n-1),...,(2)(1)=n!$. Now we can define the left canonical form which is used for a unique representation on braids to be utilized in the crypto system.
	
	\begin{definition}	
	For any $w \in B_n$ $\exists$ a unique representation called the left canonical form.
	\\$w = \Delta^u A_1 A_2 ... A_l, u \in Z', A \in \Sigma_n \backslash \{ e, \Delta \} $
	$\text{where } A_i A_{i+1} \text{is left weighted for } 1 \leq i \leq l - 1.$
	\end{definition}
	
	The braids for the crypto system are all in left canonical form for the remainder of the paper. 
	 
	\section{Hard problem associated with braids.}
	As we have seen in RSA and El-Gamal there are associated hard problems with these crypto systems. For RSA there is the factoring problem where for:
	
	\begin{itemize}
		\item p,q \text{are $2$ distinct k-bit primes.}
		\item n=pq 
		\item \text{Multiplying is computationally easy while the inverse function, factoring, is computationally hard.} 
	\end{itemize}

Similarly El-Gamal has the discrete logarithm which is the associated computationally hard problem. 
For braid group cryptography there are several computationally hard problems. One in particular that is widely used is the conjugacy search problem. Conjugacy is defined as:
\begin{definition} 
$G-\text{group. } a,x,y \in G. \text{ If }y=axa^{-1},\text{ then }y \text{ is } conjugate \text{ to }x \text{ via }a.$
\end{definition}

\noindent \textbf{Conjugacy Search Problem:} \\ $\textbf{Given }x,y \in B_n \text{ such that }y=axa^{-1} \text{ for some }a \in B_n. \\ \textbf{Find }b \in B_n \text{ such that }y=b^{-1}xb.$
\\ \\
There are several types of conjugacy search problems and one we will focus on is the Diffie-Hellman type generalized conjugacy search problem. First we need to define $2$ commuting subgroups of $B_n$. 


\begin{definition}
$LB_n,UB_n < B_n. \\ LB_n=\{ \sigma_1,...,\sigma_{[n/2]-1}  \}, UB_n=\{ \sigma_{[n/2]+1},...,\sigma_{n-1}  \} $
\end{definition}

We know use the fact by definition of a braid group that generators commute if and only if the generators do not share a common strand. Since the $\sigma_{n/2}$ generator is missing, elements in each subgroup will commute with each other such that: $a \in LB_n, b \in UB_n, ab=ba$. We can know look at the Diffie-Hellman generalized conjugacy search problem.
\\ \\
\textbf{Diffie-Hellman type Generalized Conjugacy Search Problem:} \\ $\textbf{Given }x,y_A, y_B \in B_n \text{ such that }y_A=axa^{-1} \text{ and } y_B=bxb^{-1} \text{ for some }a \in LB_n \text{ and }b \in UB_n. \\ \textbf{Find }by_Ab^{-1}=ay_Ba^{-1}=abxb^{-1}a^{-1}$

This takes advantage of the fact that we commute $a$ and $b$ which results in the following: $abxb^{-1}a^{-1}=baxa^{-1}b^{-1}$ so both Alice and Bob can decode the message. This problem is used for the key agreement between Alice and Bob.

\section{Background Info and Literature}
\textbf{Include information on what has been done. Talk about state of the art new ideas in braid group cryptography.}

\section{Braid Diffie-Hellman Key Agreement}
First we will discuss the traditional Diffie-Hellman protocol on El-Gamal then look at the key agreement for the braid version Diffie-Hellman. For the traditional Diffie-Hellman agreement the secret keys are $(a,b,g^{ab} \text{mod }p=g^{ab} \text{mod }p)$, and the public keys are $(p,g,g^a \text{mod }p,g^b \text{mod }p)$. 

\begin{definition}
$p \text{ is a prime. } g \text{ is a }\textit{primitive root } \text{of }p \text{ where }g^{p-1}\equiv 1 \text{ mod }p. $
\end{definition}

The next step in the Diffie-Hellman key exchange is for Alice and Bob to select their own keys $a,b \in \mathbb{Z}_p$  secretly and then they send $g^a \text{mod }p$ and $g^b \text{mod }p$ publicly. Then Alice and Bob exponentiate the public keys with their own secret key to obtain the same value. This is dependent on the fact that $ab=ba,  \forall a,b \in \mathbb{Z}_p$. By defining the commuting subgroups of a braid we have the tools needed to define a braid Diffie-Hellman key exchange. 
\\ \\
\textbf{Braid Diffie-Hellman key agreement system.}
\begin{enumerate}

\item \textbf{Preparation Step:} $l=\mid LB_n \mid$ and $r=\mid UB_n \mid$. Pick sufficiently complicated braid such that $x \in B_{l+r}$ which is broadcasted on a public channel. 

\item \textbf{Key Agreement:} 
	 \begin{enumerate}
	 	\item Alice chooses a random secret braid $a \in LB_n$ and sends $y_1=axa^{-1}$ to Bob.
		\item Bob chooses a random secret braid $b \in UB_n$ and sends $y_2=bxb^{-1}$ to Alice.
		\item Alice receives $y_2$ and computes the shared key $K=ay_2a^{-1}=abxb^{-1}a^{-1}$.
		\item Bob receives $y_1$ and computes the shared key $K=by_1b^{-1}=baxa^{-1}b^{-1}$.
	 \end{enumerate}
\item Therefore since for $a \in LB_n$ and $b \in UB_n$, $ab=ba$. Then
\begin{centering}
$ay_2a^{-1}=a(bxb^{-1})a^{-1}=b(axa^{-1})b^{-1}=by_1b^{-1}$. This implies that Alice and Bob both have the same shared braid.
\end{centering}
\end{enumerate}

\section{Future Direction and New Ideas}
In the braids as permutation section we mentioned that we can represent the generators of a braid as a permutation group. What about other ways to represent a braid concisely? It turns out that you can also represent generators as matrices using Dynnikov coordinates and the max-plus semiring. The Dynnikov coordinate space is $C_n=\mathbb{R}^{2n-4} \backslash \{0\}$ where $n$ is the number of strands. The idea is to represent generators by looking at how the action of the generator stretches or rotates a given vector. Therefore you can represent each generator by a matrix. This calculation is done using the max-plus semiring. We use this semiring due to the fact that it helps with computation and notation. 

\begin{definition}
The max-semiring $(\mathbb{R},\text{max},+)$ has the maximum function $\text{max}(a,b)$ as the additive operation, and the addition operation $a+b$ as the multiplicative operation. It is a semiring due to the fact that there is not an additive inverse.
\end{definition}

\noindent Here is an example that shows how to represent a generator as a matrix: For $n=3$, the possible generators are $\sigma_1^+, \sigma_1^-, \sigma_2^+, \sigma_2^-$. Also the input vector comes from the Dynnikov coordinate space: $x \in \mathbb{R}^{2n-4} = \mathbb{R}^{2(3)-4}$. For the generator $\sigma_1^+ (a,b)=(a',b')$ where 
\\ \\
$[a+b]=\text{max}(a,b), [ab]=a+b, [a/b]=a-b, \text{ and } [1]=0. \\ \\ a'=[\frac{ab}{a+1+b}], b'=[\frac{1+b}{a}],   \\ \text{ If } 0<a<b , \text{then } \sigma_1^+ (a,b)=
\begin{bmatrix}
1 & 0\\
-1 & 1
\end{bmatrix} $. We want to compare the $a$ and $b$ and obtain different matrices for several cases. Since we can represent generators as matrices we can use this to our advantage due to the fact that computers work with arrays and matrices well! This is a proposed alternative method to representing braids as permutation groups. Using results in linear algebra perhaps a more efficient system may be worth trying. 


\begin{thebibliography}{9}

\bibitem{braid_groups_intro} 
Parvez Anandam. 
\textit{Introduction to Braid Group Cryptography}. 
March 7, 2006.
 
\bibitem{new_public_crypto} 
Ki Hyoung Ko, Sang Jin Lee, Jung Hee Cheon, Jae Woo Han, Ju-sung Kang, Choonsik Park.
\textit{New Public-key Cryptosystem Using Braid Groups}.
Korea Advanced Institute of Science and Technology, Brown University, Electronics and Telecommunications Research Institute. 

\bibitem{braid_group_crypto}
David Garber.
\textit{Braid Group Cryptography}.
Department of Applied Mathematics, Holon Institute of Technology. 
 
 \bibitem{new_directions}
Whitfield Diffie, Martin Hellman. (November 1976)
\textit{New Directions in Cryptography}.
IEEE Transactions on Information Theory.


\bibitem{dynn_coordinates}
Toby Hall, S. Oyku Yurttas.
\textit{On the Topological Entropy of Families of Braids}.
\end{thebibliography}







	
\end{document}