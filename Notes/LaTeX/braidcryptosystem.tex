\documentclass{article}

\usepackage[margin=1in]{geometry}
\usepackage{amsmath}
\usepackage{amssymb}

\title{Braid Cryptosystem Notes}
\date{\today}

\begin{document}
	\maketitle
	
	\section{Braid Cryptographic System - 11/14/2019}
	
	\subsection{Braids}
	A braid is a member of the Group $B_n$.
	
	\subsection{Sub-Groups of the Braid Group}
	There are two commuting subgroups of $B_n$.
	
	\begin{align*}
		LB_n < B_n & \text{ generated by } \{ \sigma_1 , ..., \sigma_{ \left[ n/2 \right] } \}  \\
		UB_n < B_n & \text{ generated by } \{ \sigma_{ n/2 + 1 }, ..., \sigma_{n-1} \} \\
		a \in B_n & \text{ commutes w/} b \in UB_n : ab = ba 
	\end{align*}
	
	Notice how $\sigma_3$ is missing, we do this in order to be able to commute the upper and lower group.
	
	We do this using the second part of the braid definition
	%TODO: latex braid definition and reference 2nd part
	
	\subsection{Braid Cryptographic System}
	Let's define the Braid Cryptographic System.
	\begin{align*}
	n &: \text{the Braid index} \\
	l &: \text{the Canonical Index}
	\end{align*}
	
	\subsubsection{Commuter-based Key Agreement}
	
	There are many variants of the conjugacy search problem.
	
	\subsubsection{Generalized Conjugacy Search}
	Given: $x,y \in B_n \text{ s.t. } y=a^{-1}xa \text{ for some } a \in LB_n$ \\
	Find: $b \in LB_n \text{ s.t. } y = b^{-1}xb$ \\
	(note: can replace $LB_n$ w/ $UB_n$)
	
	\section*{Deliverables}
	\subsection*{11/21/2019}
	\begin{enumerate}
		\item Finish Notes (TP)
		\item Install/Demo CBraid (reference 6 of Anandam) (JL, BK, TP)
		\item Learn Cryptosystem part (RM)
	\end{enumerate}
	
\end{document}